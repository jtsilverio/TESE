  \documentclass[english,msc,numbers]{coppe}

%Added by @MyKo101, code provided by @GerbrichFerdinands
\newlength{\cslhangindent}
\setlength{\cslhangindent}{1.5em}
\newenvironment{cslreferences}%
  {\setlength{\parindent}{0pt}%
  \everypar{\setlength{\hangindent}{\cslhangindent}}\ignorespaces}%
  {\par}

\usepackage{amsmath,amssymb}
\usepackage{hyperref}
\usepackage{longtable}
\usepackage{booktabs}

\providecommand{\tightlist}{%
  \setlength{\itemsep}{0pt}\setlength{\parskip}{0pt}}

\makelosymbols
\makeloabbreviations

\begin{document}

  \title{Titulo de Tese}
  \foreigntitle{Thesis' Title}
    \author{Jefferson}{Silvério}
      \advisor{Prof.}{Gisele}{A. Oda}{D.Sc.}
    \advisor{Prof.}{Verónica}{Valentinuzzi}{Ph.D}
    \advisor{Prof.}{Patricia}{Tachinardi}{Ph.D}
  

    \examiner{Prof.}{Nome Completo do Primeiro Examinador}{D.Sc.}
    \examiner{Prof.}{Nome Completo do Segundo Examinador}{Ph.D}
    \examiner{Prof.}{Nome Completo do Terceiro Examinador}{Ph.D}
    \department{IB}
  \date{09}{2020}
    \keyword{Primeira palavra-chave}
    \keyword{Segunda palavra-chave}
    
  % Adiciona Pagina de Titulo
  \maketitle

  % Adiciona Pagina de Rosto com 
  \frontmatter
  
  %Adiciona dedicatorias
  \dedication{A alguém cujo valor é digno desta dedicatória.}
    \chapter*{Agradecimentos}
  Gostaria de agradecer a X
  
  % Adiciona Abstracts
  \begin{abstract}
  Sit urna lacus aenean euismod morbi integer mauris ligula euismod. Massa leo nunc rutrum non vulputate viverra erat aliquet torquent. Dictumst inceptos litora diam dui eu non sodales eget metus? Mollis faucibus justo class class nulla vestibulum consequat purus.

  Sit est ligula massa massa. Lectus parturient vehicula luctus nisl facilisis iaculis sagittis euismod ornare ut platea! Vestibulum et cras nostra luctus morbi cubilia et ante ornare luctus commodo facilisis nam. Lobortis ligula dictum tortor facilisis ante gravida habitasse cras laoreet. Vehicula pharetra vulputate non magna ut interdum habitant quam et class elementum arcu!

  Adipiscing nulla laoreet magna dignissim nostra phasellus lacinia elementum est id! Rutrum arcu aliquet torquent porttitor ligula eget dictumst aenean. Lacus dictumst phasellus sed lobortis leo convallis velit mi imperdiet. Ultricies convallis id vestibulum morbi rutrum tortor diam volutpat euismod montes enim cras eros luctus duis rutrum integer.

  Consectetur platea augue vitae vitae integer ad tincidunt torquent ac. Pharetra malesuada odio non lobortis dis aliquet arcu nascetur magna porttitor. Lacinia curabitur primis ligula magna sociosqu hendrerit sociosqu risus cubilia. Arcu potenti mi pellentesque nulla per varius vitae lectus pellentesque! Tempor.
  \end{abstract}
  \pagebreak
  \begin{foreignabstract}
  Sit urna lacus aenean euismod morbi integer mauris ligula euismod. Massa leo nunc rutrum non vulputate viverra erat aliquet torquent. Dictumst inceptos litora diam dui eu non sodales eget metus? Mollis faucibus justo class class nulla vestibulum consequat purus.

  Sit est ligula massa massa. Lectus parturient vehicula luctus nisl facilisis iaculis sagittis euismod ornare ut platea! Vestibulum et cras nostra luctus morbi cubilia et ante ornare luctus commodo facilisis nam. Lobortis ligula dictum tortor facilisis ante gravida habitasse cras laoreet. Vehicula pharetra vulputate non magna ut interdum habitant quam et class elementum arcu!

  Adipiscing nulla laoreet magna dignissim nostra phasellus lacinia elementum est id! Rutrum arcu aliquet torquent porttitor ligula eget dictumst aenean. Lacus dictumst phasellus sed lobortis leo convallis velit mi imperdiet. Ultricies convallis id vestibulum morbi rutrum tortor diam volutpat euismod montes enim cras eros luctus duis rutrum integer.

  Consectetur platea augue vitae vitae integer ad tincidunt torquent ac. Pharetra malesuada odio non lobortis dis aliquet arcu nascetur magna porttitor. Lacinia curabitur primis ligula magna sociosqu hendrerit sociosqu risus cubilia. Arcu potenti mi pellentesque nulla per varius vitae lectus pellentesque! Tempor.
  \end{foreignabstract}
  % Adiciona Sumário
  \tableofcontents
  
  % Adiciona Lista de Figuras
    \listoffigures
  
  % Adiciona Lista de Tabelas
    \listoftables
  
  % Adiciona Lista de Simbolos e Abreviacoes
  \printlosymbols
  \printloabbreviations

  % Adiciona Corpo da Tese
  \mainmatter
  \hypertarget{coppedownthesis_gitbook-default}{%
  \chapter{coppedown::thesis\_gitbook: default}\label{coppedownthesis_gitbook-default}}

  Placeholder

  Sit urna lacus aenean euismod morbi integer mauris ligula euismod. Massa leo nunc rutrum non vulputate viverra erat aliquet torquent. Dictumst inceptos litora diam dui eu non sodales eget metus? Mollis faucibus justo class class nulla vestibulum consequat purus.

  Sit est ligula massa massa. Lectus parturient vehicula luctus nisl facilisis iaculis sagittis euismod ornare ut platea! Vestibulum et cras nostra luctus morbi cubilia et ante ornare luctus commodo facilisis nam. Lobortis ligula dictum tortor facilisis ante gravida habitasse cras laoreet. Vehicula pharetra vulputate non magna ut interdum habitant quam et class elementum arcu!

  Adipiscing nulla laoreet magna dignissim nostra phasellus lacinia elementum est id! Rutrum arcu aliquet torquent porttitor ligula eget dictumst aenean. Lacus dictumst phasellus sed lobortis leo convallis velit mi imperdiet. Ultricies convallis id vestibulum morbi rutrum tortor diam volutpat euismod montes enim cras eros luctus duis rutrum integer.

  Consectetur platea augue vitae vitae integer ad tincidunt torquent ac. Pharetra malesuada odio non lobortis dis aliquet arcu nascetur magna porttitor. Lacinia curabitur primis ligula magna sociosqu hendrerit sociosqu risus cubilia. Arcu potenti mi pellentesque nulla per varius vitae lectus pellentesque! Tempor.

  Sit urna lacus aenean euismod morbi integer mauris ligula euismod. Massa leo nunc rutrum non vulputate viverra erat aliquet torquent. Dictumst inceptos litora diam dui eu non sodales eget metus? Mollis faucibus justo class class nulla vestibulum consequat purus.

  Sit est ligula massa massa. Lectus parturient vehicula luctus nisl facilisis iaculis sagittis euismod ornare ut platea! Vestibulum et cras nostra luctus morbi cubilia et ante ornare luctus commodo facilisis nam. Lobortis ligula dictum tortor facilisis ante gravida habitasse cras laoreet. Vehicula pharetra vulputate non magna ut interdum habitant quam et class elementum arcu!

  Adipiscing nulla laoreet magna dignissim nostra phasellus lacinia elementum est id! Rutrum arcu aliquet torquent porttitor ligula eget dictumst aenean. Lacus dictumst phasellus sed lobortis leo convallis velit mi imperdiet. Ultricies convallis id vestibulum morbi rutrum tortor diam volutpat euismod montes enim cras eros luctus duis rutrum integer.

  Consectetur platea augue vitae vitae integer ad tincidunt torquent ac. Pharetra malesuada odio non lobortis dis aliquet arcu nascetur magna porttitor. Lacinia curabitur primis ligula magna sociosqu hendrerit sociosqu risus cubilia. Arcu potenti mi pellentesque nulla per varius vitae lectus pellentesque! Tempor.

  Gostaria de agradecer a X, Y e Z.

  \hypertarget{rmd-basics}{%
  \chapter{R Markdown Basics}\label{rmd-basics}}

  Placeholder

  \hypertarget{lists}{%
  \section{Lists}\label{lists}}

  \hypertarget{line-breaks}{%
  \section{Line breaks}\label{line-breaks}}

  \hypertarget{r-chunks}{%
  \section{R chunks}\label{r-chunks}}

  \hypertarget{inline-code}{%
  \section{Inline code}\label{inline-code}}

  \hypertarget{including-plots}{%
  \section{Including plots}\label{including-plots}}

  \hypertarget{loading-and-exploring-data}{%
  \section{Loading and exploring data}\label{loading-and-exploring-data}}

  \hypertarget{additional-resources}{%
  \section{Additional resources}\label{additional-resources}}

  \hypertarget{methods}{%
  \chapter{Methods}\label{methods}}

  \hypertarget{study-site}{%
  \section{Study site}\label{study-site}}

  Field work was conducted at one site located approximately 5km away from the village of Anillaco, in the province of La Rioja, northwest of Argentina. The study site (LAT, LONG, ALTITUDE) is a relatively undisturbed natural area surrounded by the Sierra de Velasco moutain range, located within the Monte Desert biome. The Monte Desert is characterized as an open shrubland dominated by Zygophyllaceae (\emph{Larrea cuneifolia} Cav., \emph{Tricomaria usillo}), Fabaceae (\emph{Prosopis torquata}, \emph{Senna aphylla}) and Cactaceae (\emph{Trichocereus} spp, \emph{Tephrocactus} spp) (Abraham et al. \protect\hyperlink{ref-abrahamOverviewGeographyMonte2009}{2009}; Aranda-Rickert, Diez, and Marazzi \protect\hyperlink{ref-aranda-rickertExtrafloralNectarFuels2014}{2014}; Fracchia et al. \protect\hyperlink{ref-fracchiaDispersalArbuscularMycorrhizal2011}{2011}). At the study site a non-extensive survey of the plant community divided in three transects showed a dominance of the families Zygophyllaceae (\emph{Larrea cuneifolia}, \emph{Tricomaria usillo}), Poaceae (\emph{Microchloa indica}, \emph{Aristida mendocina}) and Fabaceae (\emph{Zuccagnia punctata}) (see Appendix). The climate is arid with marked seasonality.
  \begin{itemize}
  \tightlist
  \item
    sazonalidade de plantas
  \item
    temperature
  \item
    rain
  \item
    soil
  \item
    espécie
  \item
    possible predators?
  \end{itemize}
  The climate at this locality is arid with mean annual rainfall ranging from 100 to 200 mm and limited almost exclusively to the summer months (December--February) (Abraham et al., 2009). The soil is sandy and largely lacking organic matter, and the predominant vegetation is a shrubby steppe with characteristic Monte Desert flora dominated by species of Zygophyllaceae, Fabaceae and Cactaceae (Abraham et al., 2009; Fracchia et al., 2011).

  \hypertarget{study-species}{%
  \section{Study species}\label{study-species}}

  Todos os animais utilizados são indivíduos adultos (\textgreater120g) de C aff. knightii capturados em Anillaco, La Rioja, Argentina (26° 48' S; 66° 56' W; 1445 m). Os animais foram capturados em área próxima ao Centro Regional de Pesquisa Cientifica e Transferência Tecnológica de La Rioja (CRILAR). O local de coleta (S28° 47.719' W66° 53.607') possui vegetação nativa, pouca influência antrópica e nenhuma fonte de luz artificial. A região de Anillaco é localizada no Deserto do Monte, de clima semiárido, solo arenoso e com vegetação composta de arbustos, plantas rasteiras e poucas árvores (Fracchia et al, 2011).

  \hypertarget{data-collection}{%
  \section{Data collection}\label{data-collection}}

  \hypertarget{vectorial-dynamic-body-acceleration}{%
  \section{Vectorial Dynamic Body Acceleration}\label{vectorial-dynamic-body-acceleration}}

  There are multiple ways of deriving behavior and activity from an animal's acceleration record (ref). Here we chose to calculate the Vectorial Dynamic Body Acceleration (VeDBA). To calculate the VeDBA from the raw accelerometer data there are three steps. (i) Calculate the effect of the gravitational force over the device which is dependent on the animal's posture and is also known as static acceleration. The static acceleration can be estimated by calculating a moving average over the raw data. Generally, a 1 or 2-second moving average is used in this step (ref). However, there is not a consensus over the number of points to use in the moving average, which can be dependent on the study species and accelerometer recording frequency (REF??). In the case of the tucos we opted to use a 4-second (40 data points) moving average after following the method proposed by (Shepard et al. \protect\hyperlink{ref-shepardDerivationBodyMotion2008}{2008}) (see Appendix). (ii) Calculate the dynamic acceleration. The dynamic acceleration is the acceleration correspondent to the animal's movement. It can be calculated by subtracting the static acceleration from the raw acceleration for each data point. (iii) Calculate the VeDBA. The VeDBA is calculated by the vector sum of the dynamic acceleration over the device's axis.

  \[ VeDBA = \sqrt{Xd^2 + Yd^2 + Zd^2} \]

  \hypertarget{conclusion}{%
  \chapter*{Conclusion}\label{conclusion}}
  \addcontentsline{toc}{chapter}{Conclusion}

  If we don't want Conclusion to have a chapter number next to it, we can add the \texttt{\{-\}} attribute.

  \textbf{More info}

  And here's some other random info: the first paragraph after a chapter title or section head \emph{shouldn't be} indented, because indents are to tell the reader that you're starting a new paragraph. Since that's obvious after a chapter or section title, proper typesetting doesn't add an indent there.

  \hypertarget{the-first-appendix}{%
  \chapter{The First Appendix}\label{the-first-appendix}}

  Placeholder

  \hypertarget{references}{%
  \chapter*{References}\label{references}}
  \addcontentsline{toc}{chapter}{References}

  Placeholder

  \hypertarget{refs}{}
  \begin{cslreferences}
  \leavevmode\hypertarget{ref-abrahamOverviewGeographyMonte2009}{}%
  Abraham, E., H. F. del Valle, F. Roig, L. Torres, J. O. Ares, F. Coronato, and R. Godagnone. 2009. ``Overview of the Geography of the Monte Desert Biome (Argentina).'' \emph{Journal of Arid Environments} 73 (2): 144--53. \url{https://doi.org/10.1016/j.jaridenv.2008.09.028}.

  \leavevmode\hypertarget{ref-aranda-rickertExtrafloralNectarFuels2014}{}%
  Aranda-Rickert, Adriana, Patricia Diez, and Brigitte Marazzi. 2014. ``Extrafloral Nectar Fuels Ant Life in Deserts.'' \emph{AoB PLANTS} 6 (January). \url{https://doi.org/10.1093/aobpla/plu068}.

  \leavevmode\hypertarget{ref-fracchiaDispersalArbuscularMycorrhizal2011}{}%
  Fracchia, S., L. Krapovickas, A. Aranda-Rickert, and V. S. Valentinuzzi. 2011. ``Dispersal of Arbuscular Mycorrhizal Fungi and Dark Septate Endophytes by Ctenomys Cf. Knighti (Rodentia) in the Northern Monte Desert of Argentina.'' \emph{Journal of Arid Environments} 75 (11): 1016--23. \url{https://doi.org/10.1016/j.jaridenv.2011.04.034}.

  \leavevmode\hypertarget{ref-shepardDerivationBodyMotion2008}{}%
  Shepard, Elc, Rp Wilson, Lg Halsey, F Quintana, A G\a'omez Laich, Ac Gleiss, N Liebsch, Ae Myers, and B Norman. 2008. ``Derivation of Body Motion via Appropriate Smoothing of Acceleration Data.'' \emph{Aquatic Biology} 4 (December): 235--41. \url{https://doi.org/10.3354/ab00104}.
  \end{cslreferences}
\end{document}
